%!TEX root = ../report.tex
%!TeX encoding = UTF-8
%!TeX spellcheck = en_US

\chapter{Specification \& Design}\label{ch:specification-design}

The purpose of the chapter \textbf{specification \& design} section is to give to readers a clear picture of the system that will be created in terms of the required capability. A specification should tell readers what the system is required to do. The design gives them, the top-level details of how the system meets the requirements. It also points out constraints on the solution, that guided the decision making throughout the development process.

Describing what a software system does (i.e., specification) and how it does so (\ie{} design) effectively usually means describing it from multiple viewpoints. In this case, each viewpoint must convey some information about the system that other viewpoints omit. Examples of possible viewpoints include:

\begin{itemize}
  	\item the user interface
	\item the dynamic behavior of the system
	\item how data flows through the system
	\item what data types are implemented in the system
	\item what algorithms are implemented in the system
	\item the static architecture of the system, \ie{} how the code is partitioned into modules
\end{itemize}

\textbf{It is strongly recommended to make extensive use of diagrams}

It is also important to justify the design of the system, for example, by discussing the implications of the constraints on the solution and on different design choices. Then, it should gives the reasons for making the choices the team did. Typically these implications will relate to the aims of the project and to aspects of it discussed in the background section.